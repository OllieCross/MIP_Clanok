\documentclass[10pt,oneside,slovak,a4paper]{article}

\usepackage[slovak]{babel}
\usepackage[T1]{fontenc}
\usepackage[IL2]{fontenc}
\usepackage[utf8]{inputenc}
\usepackage{graphicx}
\usepackage{times}
\usepackage{pdfpages} %import pdf
\usepackage{url} %odkazy
\usepackage{hyperref} %prekliky v dokumente
\usepackage{cite} %citácie
\usepackage{verbatim} %viacriadkové komentáre
\usepackage{float} %pozicia tabulky
\usepackage{footnote} %legenda
\usepackage{tablefootnote} %legenda ku tabulke
\usepackage[square,numbers]{natbib} %číslovanie zdrojov
\usepackage{breakurl} %zalomenie linkov v literatúre
\setcitestyle{numbers} %číslovanie zdrojov
\pagestyle{plain} %jednoduché číslovanie strán
\def\UrlBreaks{\do\/\do-} %zalomenie linkov v literatúre

\title{\textbf{Využitie hier a programov virtuálnej reality ako didaktickej pomôcky v školstve}} 
\author{
	Oliver Krížovský\\[2pt]
	{\small Slovenská technická univerzita v Bratislave}\\
	{\small Fakulta informatiky a informačných technológií}\\
	{\small \texttt{xkrizovsky@stuba.sk}}\\
	{\small Semestrálny projekt v predmete Metódy inžinierskej práce, zimný semester 2022}\\
	{\small Ing.  Fedor Lehocki,  PhD.}\\
}
\date{\small 6. November 2022}

\begin{document}
\maketitle
\begin{abstract}
Na základe vlastných skúsenosti s hrami a programami vo virtuálnej realite (VR) 
som sa rozhodol venovať danej téme. Som presvedčený 
o tom, že virtuálna realita je skvelou pomôckou v školách akéhokoľvek typu, 
či už sú to základné, odborné alebo vysoké školy. Podľa môjho názoru si každá 
škola, poprípade každý učiteľ dokáže nájsť svoju tému, ktorú by lepšie vysvetlil 
alebo objasnil žiakom resp. študentom  pomocou virtuálnej reality. V mojom článku 
sa plánujem zamerať na predstavenie rôznych
didaktických~\hyperref[riesenia]{možností}
virtuálnej reality, objasniť~\hyperref[objasnenie]{problémy} v danej oblasti, preskúmať ich~\hyperref[diskusia]{implementovanie} v školskom prostredí
ako aj dokázať pozitívny~\hyperref[zhodnotenie]{výsledok použitia} 
virtuálnej reality v školstve pre lepší mentálny rozvoj žiakov,
ich vedomostí a ich elánu naučiť sa niečo nové.
\end{abstract}
\begin{center}
\emph{Kľúčové slová : Virtuálna realita, VR headsety, didaktické pomôcky, výučba pomocou technológií}
\end{center}
\section{Úvod}\label{uvod}
\qquad V tomto článku by som sa rád zaoberal problémom, ktorý podľa môjho názoru trápi
v dnešnom svete veľa žiakov a študentov. Je to fakt, že učivo, ktoré preberajú, je z ich pohľadu
nudné, nezaujímave alebo zle prezentované učiteľom. Väčšina učiteľov je takisto
nespokojná, keď sa ich žiaci na hodinách nudia alebo nedávaju pozor.
\par Niektoré z
preberaných tém na stredných školách sa tažko vysvetlujú, pretože škola buď nemá 
dostupné pomôcky alebo sa niektoré pokusy nemôžu z bezpečnostných dôvodov praktizovať 
v školskom prostredí. Implementáciu tejto technológie si viem primárne predstaviť 
najmä na predmetoch s technickým alebo prírodovedeckým zameraním,  
akými sú napríklad biológia, chémia, geografia, anatómia, fyzika a iné.
\par Takisto by som poukázal na fakt, že človek získava až 83\% informácií pomocou zraku\citep{Zdroj1}, 
tým pádom by virtuálna realita mala pomáhať získavať a uchovávať informácie oveľa lepšie ako tradičné
počúvanie výkladu učiteľa a následné písanie poznámok. VR technológie sú taktiež čoraz bežnejšie živote človeka.
Väčšina z nás už bola v kine na 3D film alebo už čo to počula o 3D tlačiarňach či 3D skenneroch\citep{Zdroj2}.
Preto si myslím že kombiácia tejto technológie a školstva by sa čoskoro mohla stať bežnejšiou a populárnejšiou.
Avšak existuje zopár problémov, ktoré brzdia tento pokrok\ldots
\section{Objasnenie problémov}\label{objasnenie}
\subsection{Finančný problém}
\par \qquad Poďla môjho názoru je jeden z významných problémov, prečo sa takáto technológia nevyužíva v praxi, jej
cena, ako môžeme vidieť v tab.1. Školy by museli nakúpiť drahé VR headsety, aby vybavili aspoň 
jednu učebňu, v ktorej by sa vyučovalo pomocou VR. Okrem headsetov by škola musela nakúpiť
aj počítače, ktoré by boli dostatočné výkonné na spustenie VR softwareu, čo je tiež finančné náročné.
Na druhej strane sa svetové technologické giganty akými su Samsung alebo Google snažia rok čo rok spraviť túto
techonógiu prístupnejšiu s využitím mobilných telefónov alebo inými spôsobmi ktoré robia túto technológiu
cenovo dostupnejšiou\citep{Zdroj3}.

\subsection{Problém so skúsenostami}
\par \qquad Ako ďalší problém treba spomenúť fakt, že učitelia sa ešte nikdy s takouto technológiou nestretli
a tým pádom by s ňou nevedeli pracovať. Z vlastnej skúsenosti viem že učiteľia na základných a stredných 
školách majú častokrát problémy zapojiť 
bežný notebook či projektor, preto si myslím že by s takouto technológiou mali veľke problémy.
To znamená, že škola by musela zamestnať technika, ktorý by takúto učebňu spravoval, čo sa takisto môže javiť ako
finančný problém pre niektoré školy.

\subsection{Problém s obsahom}
\par \qquad Jednou s ďalších prekážok pri začlenovaní VR do škôl je nedostatok kvalitného obsahu.
na svete je momentálne veľmi málo spoločností, ktoré sa zaoberajú vývojom VR aplikácií určených
priamo pre výučbu v školskom systéme. Školy by si museli sami zaplatiť vývoj softwareu, čo môže
stáť nespočetné množstvo penazí. Ostáva nám teda používať software, ktorý nie je primárne vyvynutý
na použitie v školstve, a to až kým cena za vývoj VR obsahu neklesne\citep{Zdroj4}.

\subsection{Zdravotné problémy}
\par \qquad Mnoho rodičov sa obáva o zdravie svojích detí, najmä v čase, kedy sme prekonali obdobie,
v ktorom žiaci bežne presedeli viac ako 8 hodín denne za obrazovkou kvôli dištančnej výučbe. Rodičia
maju často obavy, že takáto technológia dokáže mať nepriaznivý vplyv na zdravý vývin ich detí. Obávajú 
sa aj iných faktorov, ako je vystavenie náslinému alebo explicitnému obsahu, sociálnej izolácií
alebo príliš veľa času stráveného vo VR\citep{Zdroj4}.
\begin{table}[hbt!]
	\resizebox{\textwidth}{!}
	{
	\begin{tabular}{|c|c|c|c|c|c|}
		\hline
		\textbf{PSVR} & \textbf{HTC Vive Pro\footnotemark[1]} & \textbf{Oculus Rift\footnotemark[1]} & \textbf{Meta Quest 2\footnotemark[1]} & \textbf{Samsung Gear VR\footnotemark[2]} & \textbf{Google Cardboard\footnotemark[2]} \\ \hline
		799€&1399€&599€&499€&149€&29€\\ \hline
		\end{tabular}
	}
	\caption{Ceny populárnych VR headsetov}
	\footnotetext[1]{Použiteľné iba s výkonným PC}
\end{table}
\footnotetext[1]{Použiteľné iba s hernou konzolou}
\footnotetext[2]{Použiteľné iba s výkonným PC}
\footnotetext[3]{Použiteľné iba s mobilným telefónom}
\section{Možné riešenia}\label{riesenia}
\subsection{Vlastné riešenie}\label{riesenia:1}
\par \qquad Na zamedzenie finančného problému si stredné odborné školy a vysoké školy 
technického zamerania môžu softvér vyvinúť sami pre 
seba a pre mladších spolužiakov, či už vo forme témy bakalárskej alebo diplomovej práce
alebo aj vo voľnom čase formou projektov alebo krúžkov. 
\par Školy takisto môžu nakúpiť či 
už kartónové headsety od 
Googlu alebo iné jednoduché VR headsety ktoré využívajú mobilný telefón ako procesor a 
zároveň aj obrazovku v jednom. Do úvahy takisto prichádza možnosť headsetov, 
ktoré majú už v sebe mobilný procesor, ktorý síce nie 
je až tak výkonný, ale na edukačné účely to podľa môjho názoru bohato stačí.
\par Učitelia by takisto mali stanoviť časový limit pre používanie VR Headsetov 
, ktorý bude dostatočný na to, aby implementovanie tejto technológie
malo aj nejaký zmyslel z pohľadu na naučené učivo ale na druhej strane by tohoto času 
nemalo byť veľa na zamedzenie zdravotným a mentálnym problémom ktoré môžu vzniknúť.


\subsection{Riešenie podľa článku}\label{riesenia:2}
\par \qquad Ako ďalšie by som uviedol príklad článku, ktorý sa zaoberá
rovnakou problematikou ako tento článok. Je to štúdia ohľadom implementovania 
prednášok formou videí vo virtuálnej realite pre študentov medicíny. Najprv náhodne 
vybrali 50 študentov z ročníka ktorých následne rozdelili do dvoch skupín. Obidve 
skupiny najprv podstúpili test a následne prednášku, jedna skupina 
klasickou metódou a druhá pomocou VR. Potom čakal obidve skupiny ďalší test. Na záver 
sa výsledky obidvoch testov porovnali medzi obidvoma skupinami. Celý postup je graficky
znázornený na Obr. 1.
\par Študenti mali k dispozicií headsety typu Samsung Gear VR. Okrem cenového faktora su
najväčšími výhodami Gear VR jeho bezdrôtové vlastnosti, jednoduchosť rozhrania, ako aj 
možnosť inštalácie vlastného mobilného telefónu na headset, čo umožnuje študentom
sledovať prednášku jednoduchým umiestením ich mobilného telefónu do headset. Takýto typ headsetu
dokáže zjednodušiť
celý proces implementácie a zároveň odbúrava možné problemy, ktoré vznikajú pri prvom stretnutí
s novou a zároveň neznámou technológiou.\citep{Zdroj5}
\begin{figure}[!b]
	\begin{center}
	\includegraphics[scale=0.4]{diagram.pdf}
	\caption{Diagram procesu testovania}
	\end{center}
\end{figure}
\section{Diskusia}\label{diskusia}
\subsection{Výhody}\label{diskusia:1}
\par \qquad Najväčšia výhoda tejto technológie sú jej nekonečné možnosti využitia, kedže
v 3D softwéri dokážete dnes vytvoriť absolútne čokoľvek, či už to je reálne alebo nie. Tým pádom
dokážeme VR implementovať na mnohých predmetoch s rôznym zameraním. Virtuálna realita umožnuje študentom vzdelávať
sa pomocou skúseností, tým pádom dokáže oveľa viacej zaujať a pohltiť študenta v porovnaní s bežným
vykladom alebo prednáškou. Študenti maju následne oveľa väčšiu chuť a motiváciu naučiť sa niečo nové
\citep{zdroj6}
\subsection{Nevýhody}\label{diskusia:2}
\par \qquad Predpokladá sa, že jednou z nevýhod virtualnej reality môže byť zhoršenie zraku pri 
častom, alebo dlhodobom používani. Toto je však čisto len predpoklad, zatiaľ neexistujú
štúdie, ktoré sa touto inováciou zaoberájú dlhodobo a ktoré by tento fakt dokázali
potvrdiť alebo naopak vyvrátiť. 
\section{Zistenia a Výsledky}\label{zistenia_vysledky}
\subsection{Zistenia}\label{zistenia}
\subsection{Výsledky}\label{vysledky}
\section{Záver}\label{zaver}
\section{Zareagovanie na témy prednášok}\label{reakcia}
\subsection{Spoločenské Súvislosti}\label{reakcia:1}
\subsection{Historické Súvislosti}\label{reakcia:2}
\subsection{Technológia a Ľudia}\label{reakcia:3}

\bibliography{literatura}
\bibliographystyle{unsrt} %číslovanie bibliografie podľa výskytu v texte
\end{document}

\begin{comment}
~\ref{zaver}
~\cite{PLP-Framework}
\emph{zdôrazniť kurzívou}
\ldots{}
\footnote{Niekedy môžete potrebovať aj poznámku pod čiarou.}

\begin{itemize}
\item jedna vec
\item druhá vec
	\begin{itemize}
	\item x
	\item y
	\end{itemize}
\end{itemize}

\begin{enumerate}
\item jedna vec
\item druhá vec
	\begin{enumerate}
	\item x
	\item y
	\end{enumerate}
\end{enumerate}

\paragraph{Veľmi dôležitá poznámka.}
Niekedy je potrebné nadpisom označiť odsek. Text pokračuje hneď za nadpisom.
\end{comment}