\qquad V tomto článku by som sa rád zaoberal problémom, ktorý podľa môjho názoru trápi
v dnešnom svete veľa žiakov a študentov. Je to fakt, že učivo, ktoré preberajú, je z ich pohľadu
nudné, nezaujímave alebo zle prezentované učiteľom. Väčšina učiteľov je takisto
nespokojná, keď sa ich žiaci na hodinách nudia alebo nedávaju pozor.
\par Niektoré z
preberaných tém na stredných školách sa tažko vysvetlujú, pretože škola buď nemá 
dostupné pomôcky alebo sa niektoré pokusy nemôžu z bezpečnostných dôvodov praktizovať 
v školskom prostredí. Implementáciu tejto technológie si viem primárne predstaviť 
najmä na predmetoch s technickým alebo prírodovedeckým zameraním,  
akými sú napríklad biológia, chémia, geografia, anatómia, fyzika a iné.
\par Takisto by som poukázal na fakt, že človek získava až 83\% informácií pomocou zraku\citep{Zdroj1}, 
tým pádom by virtuálna realita mala pomáhať získavať a uchovávať informácie oveľa lepšie ako tradičné
počúvanie výkladu učiteľa a následné písanie poznámok. VR technológie sú taktiež čoraz bežnejšie živote človeka.
Väčšina z nás už bola v kine na 3D film alebo už čo to počula o 3D tlačiarňach či 3D skenneroch\citep{Zdroj2}.
Preto si myslím že kombiácia tejto technológie a školstva by sa čoskoro mohla stať bežnejšiou a populárnejšiou.
Avšak existuje zopár problémov, ktoré brzdia tento pokrok\ldots