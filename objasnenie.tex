\subsection{Finančný problém}
\par \qquad Poďla môjho názoru je jeden z významných problémov, prečo sa takáto technológia nevyužíva v praxi, jej
cena, ako môžeme vidieť v tab.1. Školy by museli nakúpiť drahé VR headsety, aby vybavili aspoň 
jednu učebňu, v ktorej by sa vyučovalo pomocou VR. Okrem headsetov by škola musela nakúpiť
aj počítače, ktoré by boli dostatočné výkonné na spustenie VR softwareu, čo je tiež finančné náročné.
Na druhej strane sa svetové technologické giganty akými su Samsung alebo Google snažia rok čo rok spraviť túto
techonógiu prístupnejšiu s využitím mobilných telefónov alebo inými spôsobmi ktoré robia túto technológiu
cenovo dostupnejšiou\citep{Zdroj3}.

\subsection{Problém so skúsenostami}
\par \qquad Ako ďalší problém treba spomenúť fakt, že učitelia sa ešte nikdy s takouto technológiou nestretli
a tým pádom by s ňou nevedeli pracovať. Z vlastnej skúsenosti viem že učiteľia na základných a stredných 
školách majú častokrát problémy zapojiť 
bežný notebook či projektor, preto si myslím že by s takouto technológiou mali veľke problémy.
To znamená, že škola by musela zamestnať technika, ktorý by takúto učebňu spravoval, čo sa takisto môže javiť ako
finančný problém pre niektoré školy.

\subsection{Problém s obsahom}
\par \qquad Jednou s ďalších prekážok pri začlenovaní VR do škôl je nedostatok kvalitného obsahu.
na svete je momentálne veľmi málo spoločností, ktoré sa zaoberajú vývojom VR aplikácií určených
priamo pre výučbu v školskom systéme. Školy by si museli sami zaplatiť vývoj softwareu, čo môže
stáť nespočetné množstvo penazí. Ostáva nám teda používať software, ktorý nie je primárne vyvynutý
na použitie v školstve, a to až kým cena za vývoj VR obsahu neklesne\citep{Zdroj4}.

\subsection{Zdravotné problémy}
\par \qquad Mnoho rodičov sa obáva o zdravie svojích detí, najmä v čase, kedy sme prekonali obdobie,
v ktorom žiaci bežne presedeli viac ako 8 hodín denne za obrazovkou kvôli dištančnej výučbe. Rodičia
maju často obavy, že takáto technológia dokáže mať nepriaznivý vplyv na zdravý vývin ich detí. Obávajú 
sa aj iných faktorov, ako je vystavenie náslinému alebo explicitnému obsahu, sociálnej izolácií
alebo príliš veľa času stráveného vo VR\citep{Zdroj4}.